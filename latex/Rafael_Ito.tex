
%-------------------------------------------------------------------------------
%                             ADDITIONAL PACKAGES
%-------------------------------------------------------------------------------
\documentclass[
	a4paper,
%	showframes,
%	vline=2.2em,
%	maincolor=cvgreen,
%	sectioncolor=red,
%	subsectioncolor=orange,
%	itemtextcolor=black!80,
%	sidebarwidth=0.4\paperwidth,
%	topbottommargin=0.03\paperheight,
%	leftrightmargin=20pt,
%	profilepicsize=4.5cm,
%	profilepicborderwidth=3.5pt,
%	profilepicstyle=profilecircle,
%	profilepiczoom=1.0,
%	profilepicxshift=0mm,
%	profilepicyshift=0mm,
% profilepicrounding=1.0cm,
]{fortysecondscv}

% include fonts
\pdfmapfile{=ClearSans.map}
\pdfmapfile{=fontawesome.map}

% include and use scalable cm-super fonts
\usepackage[T1]{fontenc}
\usepackage{lmodern}

% improve word spacing and hyphenation
\usepackage{microtype}
\usepackage{ragged2e}

% take care of proper font encoding
\ifxetexorluatex
	\usepackage{fontspec}
	\defaultfontfeatures{Ligatures=TeX}
%	\newfontfamily\headingfont[Path = fonts/]{segoeuib.ttf} % local font
\else
	\usepackage[utf8]{inputenc}
	\usepackage[T1]{fontenc}
%	\usepackage[sfdefault]{noto} % use noto google font
\fi

% enable mathematical syntax for some symbols like \varnothing
\usepackage{amssymb}

% bubble diagram configuration
%\usepackage{smartdiagram}
%\smartdiagramset{
%	% default font size is \large, so adjust to harmonize with sidebar layout
%	bubble center node font = \footnotesize,
%	bubble node font = \footnotesize,
%	% default: 4cm/2.5cm; make minimum diameter relative to sidebar size
%	bubble center node size = 0.4\sidebartextwidth,
%	bubble node size = 0.25\sidebartextwidth,
%	distance center/other bubbles = 1.5em,
%	% set center bubble color
%	bubble center node color = maincolor!70,
%	% define the list of colors usable in the diagram
%	set color list = {maincolor!10, maincolor!40,
%	maincolor!20, maincolor!60, maincolor!35},
%	% sets the opacity at which the bubbles are shown
%	bubble fill opacity = 0.8,
%}


%-------------------------------------------------------------------------------
%                            PERSONAL INFORMATION
%-------------------------------------------------------------------------------
%% mandatory information
% your name
\cvname{Panda Bear}
% job title/career
\cvjobtitle{Panda Scientist,\\[0.2em] Panda of the Year}

%% optional information
% profile picture
\cvprofilepic{../figures/profile.png}

% NOTE: ordering in sidebar will mimic the following order
% date of birth
\cvbirthday{\today}
% short address/location, use \newline if more than 1 line is required
\cvaddress{Park Ave.~1, 555 555 B-Woods}
% phone number
\cvphone{+86 555 555 555}
% personal website
\cvsite{https://pandascience.net}
% email address
\cvmail{panda@bamboo.cn}
% pgp key
\cvkey{4096R/FF00FF00}{0xAABBCCDDFF00FF00}
% any other custom entry
\cvcustomdata{\faFlag}{Chinese}

%-------------------------------------------------------------------------------
%                              SIDEBAR 1st PAGE
%-------------------------------------------------------------------------------
% add more profile sections to sidebar on first page
\addtofrontsidebar{
	% include gosquare national flags from https://github.com/gosquared/flags;
	% naming according to ISO 3166-1 alpha-2 country codes
    \graphicspath{{../figures/flags/shiny/}}

	% social network accounts incl. proper hyperlinks
	\profilesection{Social Network}
		\begin{icontable}{2.5em}{1em}
			\social{\aiOverleafSquare}
				{https://de.overleaf.com/latex/templates/forty-seconds-cv/pztcktmyngsk}
				{Overleaf Template Link}
			\social{\faGithub}
				{https://github.com/PandaScience/FortySecondsCV}
				{Github Project Page Link}
		\end{icontable}

	\profilesection{Languages}
		\pointskill{\flag{CN.png}}{Chinese}{5}
		\pointskill{\flag{DE.png}}{German}{3}
	\pointskill{\flag{GB.png}}{English}{3}
	\pointskill{\flag{FR.png}}{French}{3}

	\profilesection{Hard Skills}
		\skill{\faBalanceScale}{Sleeping almost all day}
		\skill{\faSitemap}{Eating a lot bamboo sprouts}
		\skill{\faGraduationCap}{Relaxing rest of the day}

	\profilesection{Soft Skills}
		\pointskill{\faHome}{Looking Cute}{4}[4]
			\skill[1.8em]{\faCompress}{No need to specify further}
		\pointskill{\faChild}{Chillin' hard}{3}[4]
			\skill[1.8em]{\faCompress}{On a tree}
			\skill[1.8em]{\faCompress}{On the grass}
}


%-------------------------------------------------------------------------------
%                              SIDEBAR 2nd PAGE
%-------------------------------------------------------------------------------
\addtobacksidebar{
	\profilesection{About Me}
	\aboutme{
		The giant panda is a terrestrial animal and primarily spends its life
		roaming and feeding in the bamboo forests of the Qinling Mountains and in
		the hilly province of Sichuan.
	}

	\profilesection{Diagrams}
	\chartlabel{Bubble Diagram}
	\begin{figure}\centering
		\smartdiagram[bubble diagram]{
			\textcolor{white}{\textbf{Being a}} \\
			\textcolor{white}{\textbf{Panda}}, % center bubble
			\textcolor{black!90}{Eating},
			\textcolor{black!90}{Sleeping},
			\textcolor{black!90}{Rolling},
			\textcolor{black!90}{Playing},
			\textcolor{black!90}{Chilling}
		}
	\end{figure}

	\chartlabel{Wheel Chart}

	\wheelchart{4em}{2em}{%
	20/3em/maincolor!50/Chill,
	15/3em/maincolor!15/Play,
	30/4em/maincolor!40/Sleep,
	20/3em/maincolor!20/Eat
	}

	\profilesection{Barskills}
	\barskill{\faSkyatlas}{Wearing asian rice hats}{60}
	\barskill{\faImage}{Playing Chess}{30}
	\barskill{\faMusic}{Playing the bamboo flute}{50}

	\profilesection{Memberships}
	\begin{memberships}
        \membership[4em]{../figures/logo.png}{PandaScience.net}
        \membership[4em]{../figures/logo.png}{Some long text spanning over more than
			only one line}
            \membership[4em]{../figures/logo.png}{\rule{\linewidth}{1pt}}
	\end{memberships}
}


%-------------------------------------------------------------------------------
%                         TABLE ENTRIES RIGHT COLUMN
%-------------------------------------------------------------------------------
\begin{document}

\makefrontsidebar

\cvsection{Working Experience}
\begin{cvtable}[3]
	\cvitem{currently}{CEO The Panda Way}{Start Up}{Chief executive officer, Head
		developer and yoga ambassador of 'The Panda Way' - A company from pandas
		for pandas.}
	\cvitem{2015 -- 2018}{Panda Scientist}{Bamboo University}{
		Reasearching the impact of adequate bamboo nutrition compared to
		conventional feeding methods.}
	\cvitem{2010 -- 2015}{Bamboo Broker}{Stock Exchange}{Continuously achieving
		better bamboo bangs for the buck.}
\end{cvtable}


\cvsection{Education}
\cvsubsection{Postgraduate Training}
\begin{cvtable}[1.5]
	\cvitem{2009 -- 2010}{Post-Doc Panda Studies}{Panda Academy}
		{In-depth studies on the impact of bamboo nutrition for young pandas and
		its relation to relaxing, sleeping and snoozing parts of the day.}
	\cvitem{2008 -- 2009}{Research Stay Europe}{European Panda Labs}
		{Spending one year abroad teaching european panda facilities about the
		newest findings and research in the field of asian rice hat covers and
		applications for bamboo as a material.}
\end{cvtable}


\cvsubsection{Study}
\begin{cvtable}[1.5]
	\cvitem{2006 -- 2008}{Master Studies Panda Science}{Panda Academy}
		{Focus: Advanced rice hat studies and nouveau rain-reflecting cover
		materials.}
	\cvitem{}{Master Theses ($\varnothing\,	1,0$)}{Asian Rice Hat Institute}
		{Impact on solar radiation onto rice hat cover materials with special
		attention to water resistance.}
	\cvitem{2003 -- 2006}{Bachelor Studies PandaScience}{Panda Academy}
		{Focus: Bamboo morphology and its usage in different craftmanships.}
	\cvitem{}{Bachelor Theses ($\varnothing\,	1,0$)}{Bamboo Institute}
		{The bambo flute: An underestimated instrument in orchestras?}
\end{cvtable}

\cvsection{Publications}
\begin{cvtable}
	\cvpubitem{Cooking: 100 recipes for lazy Pandas}{Me and My Panda Friends}
		{Panda's Culinary World}{2010}
	\cvpubitem{Pandastasia}{Still Me}{Bamboo Books Assoc.}{2005}
\end{cvtable}


\cvsection{Awards}
\begin{cvtable}
	\cvitem{2010 -- now}{Panda of the Year}{Panda World Forum}{}
	\cvitem{2005 -- now}{Face of World Wide Fund for Nature}{WWF}{}
	\cvitem{2000}{Winner of Bamboo Sprouts Eating Contest}{Bamboo Society}{}
\end{cvtable}


\cvsection{Extra-Curricular Activities}
\begin{cvtable}
	\cvitemshort{Relaxing}{Master the fine art of relaxing everywhere}
	\cvitemshort{Music}{Playing the bamboo flute in the 1st Panda Orchestra}
	\cvitemshort{Education}{Teaching young pandas to be more panda-like}
\end{cvtable}


\newpage
\makebacksidebar


\cvsection{section}
\cvsubsection{Subsection}
\begin{cvtable}
	\cvitem{<dates>}{<cv-item title>}{<location>}{<optional: description>}
\end{cvtable}

\cvsection{cvitem}
\cvsubsection{Multi-line with longer description}
\begin{cvtable}
	\cvitem{date}{Description}{location}{Some longer and more detailed
		description, that takes two lines of space instead of only one.}
	\cvitem{date}{Description}{location}{Some longer and more detailed
		description, that takes two lines of space instead of only one.}
	\cvitem{date}{Description}{location}{Some longer and more detailed
		description, that takes two lines of space instead of only one.}
\end{cvtable}

\cvsubsection{One-line without description}
\begin{cvtable}
	\cvitem{Award}{One-line description}{Sponsor}{}
	\cvitem{Award}{One-line description}{Sponsor}{}
	\cvitem{Award}{One-line description}{Sponsor}{}
\end{cvtable}

\cvsection{cvitemshort}
\cvsubsection{One-line}
\begin{cvtable}
	\cvitemshort{Key}{Some further description}
	\cvitemshort{Key}{Some further description}
	\cvitemshort{Key}{Some further description}
\end{cvtable}

\cvsubsection{Multi-line with longer description}
\begin{cvtable}
	\cvitemshort{Key}{Some further description. Can fill even more than
		only one single line while still keeping the correct indendation level.}
	\cvitemshort{Key}{Some further description. Can fill even more than
		only one single line while still keeping the correct indendation level.}
	\cvitemshort{Key}{Some further description. Can fill even more than
		only one single line while still keeping the correct indendation level.}
\end{cvtable}

\cvsection{cvpubitem}
\begin{cvtable}
	\cvpubitem{Publication title}{Authors}{Journal}{Year}
	\cvpubitem{Publication title}{Authors}{Journal}{Year}
	\cvpubitem{Publication title that is spanning over multiple lines and still
		does not look too bad}{Authors}{Journal}{Year}
\end{cvtable}

\cvsignature

\end{document}
